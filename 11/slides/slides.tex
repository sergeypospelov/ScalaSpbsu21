\documentclass[aspectratio=169]{beamer}

% SETUP =====================================
\usepackage[T1,T2A]{fontenc}
\usepackage[utf8]{inputenc}
\usepackage[russian]{babel}
\usepackage{listings}
\usepackage{array}
\usepackage{amssymb}
\usepackage{pifont}
\usepackage{minted}
\usepackage{pgf-pie}
\usepackage{fancyvrb}
\usepackage{../../beamerthemeslidesgeneric}
% SETUP =====================================

\title{Scala Ecosystem: Cats Effect}
\author{Mikhail Mutcianko, Alexey Shcherbakov}
\institute{СПБгУ, СП}
\date{22 апреля 2021}

\begin{document}

\frame{\titlepage}

% refferentail transparency
\begin{frame}{Effect}
  In functional programming an \textit{effect} is a context in which your computation operates:
  \begin{itemize}
    \item \texttt{Option} models the absence of value
    \item \texttt{Try} models the possibility of failure
    \item \texttt{Future} models the asynchronicity of a computation
    \item \ldots
  \end{itemize}
  \bigskip
  Side effects thereby denote computational leaks that escape your effect, or context
\end{frame}

\begin{frame}[fragile]{Referential Transparency}
 An expression is said to be \textit{referential transparent} when it can be replaced by its value without
 changing the program
 \bigskip
\begin{minted}{scala}
def sum(a: Int, b: Int): Int = a + b

assertEquals(5, sum(2,3))
\end{minted}
\end{frame}

% basic principle: don't run side effects - IO monad
\begin{frame}[fragile]{Principle}
\begin{center}
How to make code that performs side-effects referentially transparent?\\
\pause
  \Large Not to run it
\end{center}
\pause
\texttt{cats.effect.IO} does exactly this: it wraps a computation but doesn’t run it

\begin{minted}{scala}
val n = IO {  // doesn't print anything
  print("n")
  3
}
\end{minted}
\end{frame}

\begin{frame}[fragile]{IO basics}
You run an \texttt{IO} by calling one of the “unsafe” method that it provides:
\begin{itemize}
  \item \texttt{unsafeRunSync} runs the program synchronously
  \item \texttt{unsafeRunTimed} runs the program synchronously but abort after a specified timeout 
  \item \texttt{unsafeRunAsync} runs the program asynchronously and execute the specified callback when done
  \item \texttt{unsafeToFuture} runs the program and produces the result as a Scala Future 
  \item \ldots
\end{itemize}
\end{frame}

% conditional running
\begin{frame}[fragile]{Conditional running}
By not running the code straight away we get additional capabilities. E.g. it’s possible to decide
to run a computation after declaring it:
\bigskip
\begin{minted}{scala}
val isWeekday = true
for {
   _ <- IO(println("Working")).whenA(isWeekday)
   _ <- IO(println("Offwork")).unlessA(isWeekday)
} yield ()
\end{minted}
\end{frame}

% async running
\begin{frame}[fragile]{Async IO}
When you need to interact with remote systems and in this case you need an \texttt{asynchronous IO}.
An \texttt{asynchronous IO} is created by passing a callback that is invoked when the computation completes.
\bigskip
\begin{minted}{scala}
def fromCompletableFuture[A](f: => CompetableFuture[A]): IO[A] =
  IO.async { callback =>
    f.whenComplete { (res: A, error: Throwable) =>
      if (error == null) callback(Right(res))
      else callback(Left(error))
    } 
  }
\end{minted}
\end{frame}

% brackets
\begin{frame}[fragile]{Brackets}
  IO being lazy it’s possible to make sure all the resources used in the computation are released
  properly no matter the outcome of the computation. Think of it as a \texttt{try~/~catch~/~finally}.
  In cats-effect this is called \texttt{Bracket}
  \bigskip
  \begin{minted}{scala}
    // acquire the resource
    IO(new Socket("hostname", 12345)).bracket { socket =>
      // use the socket here
    } { socket =>
      // release block
      socket.close()
    }
  \end{minted}
\end{frame}

% resources
\begin{frame}[fragile]{Resources}
  \texttt{Resource} builds on top of \texttt{Bracket}. It allows to acquire resources (and making sure they are
  released properly) before running your computation. Moreover \texttt{Resource} is composable so that you
  can acquire all your resources at once.
  \begin{minted}{scala}
def acquire(s: String) = IO(println(s"Acquire $s")) *> IO.pure(s)
def release(s: String) = IO(println(s"Releasing $s"))
val resources = for {
   a <- Resource.make(acquire("A"))(release("A"))
   b <- Resource.make(acquire("B"))(release("B"))
} yield (a ,b)

resources.use { case (a, b) => // use a and b
} // release code is automatically invoked when computation finishes
  \end{minted}
\end{frame}

% cancellation
\begin{frame}[fragile]{Cancellation}
Async \texttt{IO} takes \texttt{Callback => Unit}\\
To be able to cancel, we need to somehow pass another \texttt{IO} that's able to cancel the computation
\bigskip
\begin{minted}{scala}
def cancelable[A](k: (Either[Throwable, A] => Unit) => IO[Unit]): IO[A] = ???
def fromCompletableFuture[A](f: => CompletableFuture[A]): IO[A] =
  IO.cancellable { callback =>
    f.whenComplete(res: A, error: Throwable) =>
      if (error == null) callback(Right(res))
      else               callback(Left(error))
    IO(f.cancel(true)) // cancelling IO
  }
\end{minted}
\end{frame}

% fibers

\end{document}

